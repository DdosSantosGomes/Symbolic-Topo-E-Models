%%%%%%%%%%%%%%%%%%%%%%%%%%%%%%%%%%%%%%%%%
% Beamer Presentation
% LaTeX Template
% Version 2.0 (March 8, 2022)
%
% This template originates from:
% https://www.LaTeXTemplates.com
%
% Author:
% Vel (vel@latextemplates.com)
%
% License:
% CC BY-NC-SA 4.0 (https://creativecommons.org/licenses/by-nc-sa/4.0/)
%
%%%%%%%%%%%%%%%%%%%%%%%%%%%%%%%%%%%%%%%%%

%----------------------------------------------------------------------------------------
%    PACKAGES AND OTHER DOCUMENT CONFIGURATIONS
%----------------------------------------------------------------------------------------

\documentclass[
	11pt, % Set the default font size, options include: 8pt, 9pt, 10pt, 11pt, 12pt, 14pt, 17pt, 20pt
	%t, % Uncomment to vertically align all slide content to the top of the slide, rather than the default centered
	aspectratio=1610, % Uncomment to set the aspect ratio to a 16:9 ratio which matches the aspect ratio of 1080p and 4K screens and projectors
]{beamer}

\graphicspath{{images/}{./}} % Specifies where to look for included images (trailing slash required)

\usepackage{booktabs} % Allows the use of \toprule, \midrule and \bottomrule for better rules in tables

%----------------------------------------------------------------------------------------
%    MY PACKAGES AND DEFINITIONS
%----------------------------------------------------------------------------------------

\usepackage{
    adjustbox,
    amsfonts,
    amsmath,
    amssymb,
    array,
    cancel,
    fancyhdr,
    graphicx,
    mathrsfs,
    mathtools,
    mdframed,
    multicol,
    nicefrac,
    prftree,
    quiver,
    scalerel,
    stmaryrd,
    tikz-cd,
    trimclip,
    wrapfig,
}

% Meta-information
\def\TODO{\textcolor{red}{\textbf{Finish me.}}}
\def\FIXME{\textcolor{red}{\textbf{Fix me.}}}
\def\reff{\textcolor{red}{\textbf{?}}}
% Fonts
\def\Bl{\Bigl}
\def\Br{\Bigr}
% Sets
\def\contains{\ni}
\def\notcontains{\notbin{\ni}}
\def\empty{\varnothing}
\def\sub{\subseteq}
\def\sup{\supseteq}
\def\notsub{\notbin{\sub}}
\def\notsup{\notbin{\sup}}
\def\union{\cup}
\def\inter{\cap}
\def\bigunion{\bigcup}
\def\biginter{\bigcap}
\def\dunion{\sqcup}
\newcommand{\pset}[1]{\wp(#1)}
\newcommand{\set}[1]{\{#1\}}
\newcommand{\indset}[2]{\set{#1}_{#2}}
\newcommand{\bindset}[3]{\set{#1}_{#2}^{#3}}
\newcommand{\comp}[2]{\{ #1 \mid #2 \}}
\newcommand{\compin}[3]{\{ #1 \in #2 \mid #3 \}}
\newcommand{\compsub}[3]{\{ #1 \sub #2 \mid #3 \}}
\newcommand{\quo}[2]{{#1}_{\nicefrac{}{#2}}}
% Functions
\def\too{\longrightarrow}
\def\mapstoo{\longmapsto}
\def\after{\circ}
\def\inject{\xhookrightarrow{}}
% General logic
\def\imp{\rightarrow}
\def\bigland{\bigwedge}
\def\biglor{\bigvee}
\def\refutes{\notbin{\models}}
\def\seqimp{\Rightarrow}
\def\iso{\cong}
\def\notiso{\notbin{\simeq}}
\def\substruct{\mathbin{\trianglelefteqslant}}
\def\notsubstruct{\notbin{\substruct}}
\def\proves{\vdash}
\newcommand{\provable}[2]{\proves^{#1}_{#2}}
% Modal logic
\def\Dia{\Diamond}
\def\HalfDiamondLeft{\ensuremath{\clipbox{0pt 0pt 4pt 0pt}{$\Diamond$}}}
\def\HalfDiamondRight{\ensuremath{\clipbox{4pt 0pt 0pt 0pt}{$\Diamond$}}}
\def\HalfBoxLeft{\ensuremath{\clipbox{0pt 0pt 4pt 0pt}{$\square$}}}
\def\HalfBoxRight{\ensuremath{\clipbox{4pt 0pt 0pt 0pt}{$\square$}}}
\def\Dox{\HalfDiamondLeft\raisebox{-1.4pt}{\scalebox{1.2}{\HalfBoxRight}}}
\def\Bia{\raisebox{-1.4pt}{\scalebox{1.2}{\HalfBoxLeft}}\HalfDiamondRight}
\def\mcbox{\mathrlap{\Box}{\hspace{2.3pt}\raisebox{1.8pt}{\scalebox{0.8}{$*$}}}\hspace{2.2pt}}
\def\mcdia{\mathrlap{\Diamond}{\hspace{1.6pt}\raisebox{1.2pt}{\scalebox{0.8}{$*$}}}\hspace{1.5pt}}
% Orders
\def\leq{\leqslant}
\def\geq{\geqslant}
\def\notleq{\notbin{\leq}}
\def\notgeq{\notbin{\geq}}
\def\notlt{\notbin{<}}
\def\notgt{\notbin{>}}
\def\join{\vee}
\def\meet{\wedge}
\def\bigjoin{\bigvee}
\def\bigmeet{\bigwedge}
\newcommand{\upset}[1]{{\uparrow}#1}
\newcommand{\downset}[1]{{\downarrow}#1}
\newcommand{\Upset}[1]{{\Uparrow}#1}
\newcommand{\Downset}[1]{{\Downarrow}#1}
% Numbers
\def\NN{\mathbb{N}}
\def\QQ{\mathbb{Q}}
\def\RR{\mathbb{R}}
% Topology
\newcommand{\interior}[1]{\text{int}(#1)}
\newcommand{\closure}[1]{\overline{#1}}
% Proof-writing
\def\subsubqed{\hfill $\checkmark$}
\def\subqed{\hfill $\square$}
\def\qed{\hfill $\blacksquare$}
\def\mimp{\Rightarrow}
\def\miff{\Leftrightarrow}
\def\defiff{\ :\Longleftrightarrow}
\def\awtbs{as was to be shown}
\def\wlogmimp{\xRightarrow{\text{WLOG}}}
\def\LHS{\text{LHS}}
\def\RHS{\text{RHS}}
\def\Nte{\emph{Note - }}
\newcommand{\Rmk}[1]{\emph{Remark #1 -}}
\newcommand{\Def}[2]{\textbf{Definition #1} (#2) -}
\newcommand{\Exm}[1]{\textbf{Example #1} -}
\newcommand{\Thm}[1]{\textbf{Theorem #1} -}
\newcommand{\Lem}[1]{\textbf{Lemma #1} -}
\newcommand{\Fct}[1]{\textbf{Fact #1} -}
\newcommand{\Obs}[1]{\textbf{Observation #1} -}
\def\Proof{\emph{Proof - }}
% Project-specific
\def\XX{\mathbf{X}}
\def\SFour{\mathbf{S4}}
\def\Topo{\mathbf{Topo}}
\def\Alx{\mathbf{Alx}}
\def\Pre{\mathbf{Pre}}
\def\Prop{\mathbf{Prop}}
\newcommand{\link}[2]{\href{#1}{{\color{blue}\underline{#2}}}}
\def\M{\mathfrak{M}}
\def\Up{\mathtt{Up}}
\def\RX{R_{\XX}}


%----------------------------------------------------------------------------------------
%    SELECT LAYOUT THEME
%----------------------------------------------------------------------------------------

% Beamer comes with a number of default layout themes which change the colours and layouts of slides. Below is a list of all themes available, uncomment each in turn to see what they look like.

%\usetheme{default}
%\usetheme{AnnArbor}
%\usetheme{Antibes}  % No footer
%\usetheme{Bergen}
%\usetheme{Berkeley}
%\usetheme{Berlin}
%\usetheme{Boadilla}
%\usetheme{CambridgeUS}  % Lots of whitespace
%\usetheme{Copenhagen}  % No page numbers
%\usetheme{Darmstadt}
%\usetheme{Dresden}
%\usetheme{Frankfurt}
%\usetheme{Goettingen}
%\usetheme{Hannover}
%\usetheme{Ilmenau}
%\usetheme{JuanLesPins}
%\usetheme{Luebeck}  % Same as Copenhagen ?
\usetheme{Madrid}  % Favourite
%\usetheme{Malmoe}  % Cleaner than Copenhagen
%\usetheme{Marburg}
%\usetheme{Montpellier}
%\usetheme{PaloAlto}
%\usetheme{Pittsburgh}
%\usetheme{Rochester}
%\usetheme{Singapore}
%\usetheme{Szeged}
%\usetheme{Warsaw}

%----------------------------------------------------------------------------------------
%    SELECT COLOR THEME
%----------------------------------------------------------------------------------------

% Beamer comes with a number of color themes that can be applied to any layout theme to change its colours. Uncomment each of these in turn to see how they change the colours of your selected layout theme.

%\usecolortheme{albatross}
%\usecolortheme{beaver}
%\usecolortheme{beetle}
%\usecolortheme{crane}
%\usecolortheme{dolphin}
%\usecolortheme{dove}
%\usecolortheme{fly}
\usecolortheme{lily}  % Favourite
%\usecolortheme{monarca}
%\usecolortheme{seagull}
%\usecolortheme{seahorse}
%\usecolortheme{spruce}
%\usecolortheme{whale}
%\usecolortheme{wolverine}

%----------------------------------------------------------------------------------------
%    SELECT FONT THEME & FONTS
%----------------------------------------------------------------------------------------

% Beamer comes with several font themes to easily change the fonts used in various parts of the presentation. Review the comments beside each one to decide if you would like to use it. Note that additional options can be specified for several of these font themes, consult the beamer documentation for more information.

\usefonttheme{default} % Typeset using the default sans serif font  % Favourite
%\usefonttheme{serif} % Typeset using the default serif font (make sure a sans font isn't being set as the default font if you use this option!)
%\usefonttheme{structurebold} % Typeset important structure text (titles, headlines, footlines, sidebar, etc) in bold
%\usefonttheme{structureitalicserif} % Typeset important structure text (titles, headlines, footlines, sidebar, etc) in italic serif
%\usefonttheme{structuresmallcapsserif} % Typeset important structure text (titles, headlines, footlines, sidebar, etc) in small caps serif

%------------------------------------------------

%\usepackage{mathptmx} % Use the Times font for serif text
\usepackage{palatino} % Use the Palatino font for serif text  % Favourite

%\usepackage{helvet} % Use the Helvetica font for sans serif text
%\usepackage[default]{opensans} % Use the Open Sans font for sans serif text
%\usepackage[default]{FiraSans} % Use the Fira Sans font for sans serif text
\usepackage[default]{lato} % Use the Lato font for sans serif text  % Favourite

%----------------------------------------------------------------------------------------
%    SELECT INNER THEME
%----------------------------------------------------------------------------------------

% Inner themes change the styling of internal slide elements, for example: bullet points, blocks, bibliography entries, title pages, theorems, etc. Uncomment each theme in turn to see what changes it makes to your presentation.

\useinnertheme{circles}  % Favourite
%\useinnertheme{rectangles}

%----------------------------------------------------------------------------------------
%    SELECT OUTER THEME
%----------------------------------------------------------------------------------------

% Outer themes change the overall layout of slides, such as: header and footer lines, sidebars and slide titles. Uncomment each theme in turn to see what changes it makes to your presentation.

%\useoutertheme{default}
%\useoutertheme{infolines}
\useoutertheme{miniframes}
%\useoutertheme{sidebar}
%\useoutertheme{split}  % Favourite
%\useoutertheme{tree}

%\setbeamertemplate{footline} % Uncomment this line to remove the footer line in all slides
%\setbeamertemplate{footline}[page number] % Uncomment this line to replace the footer line in all slides with a simple slide count

\setbeamertemplate{navigation symbols}{} % Uncomment this line to remove the navigation symbols from the bottom of all slides

%----------------------------------------------------------------------------------------
%    PRESENTATION INFORMATION
%----------------------------------------------------------------------------------------

\title[Topomodels]{Topomodels} % The short title in the optional parameter appears at the bottom of every slide, the full title in the main parameter is only on the title page

\subtitle{An implementation of topological semantics \\ for modal logic in Haskell} % Presentation subtitle, remove this command if a subtitle isn't required

\author[David Álvarez Lombardi \& Paulius Sakaisgiris]{David Álvarez Lombardi \and Paulius Sakaisgiris} % Presenter name(s), the optional parameter can contain a shortened version to appear on the bottom of every slide, while the main parameter will appear on the title slide

\institute[ILLC]{Institute for Logic, Language, and Computation \\ \smallskip University of Amsterdam} % Your institution, the optional parameter can be used for the institution shorthand and will appear on the bottom of every slide after author names, while the required parameter is used on the title slide and can include your email address or additional information on separate lines

\date[Friday, 2 June 2023]{Functional Programming\\ \smallskip Friday, 2 June 2023} % Presentation date or conference/meeting name, the optional parameter can contain a shortened version to appear on the bottom of every slide, while the required parameter value is output to the title slide

%----------------------------------------------------------------------------------------

\begin{document}

%----------------------------------------------------------------------------------------
%    TITLE SLIDE
%----------------------------------------------------------------------------------------

\begin{frame}
	\titlepage % Output the title slide, automatically created using the text entered in the PRESENTATION INFORMATION block above
\end{frame}

%----------------------------------------------------------------------------------------
%    TABLE OF CONTENTS SLIDE
%----------------------------------------------------------------------------------------

% The table of contents outputs the sections and subsections that appear in your presentation, specified with the standard \section and \subsection commands. You may either display all sections and subsections on one slide with \tableofcontents, or display each section at a time on subsequent slides with \tableofcontents[pausesections]. The latter is useful if you want to step through each section and mention what you will discuss.

\begin{frame}
	\frametitle{Presentation Overview} % Slide title, remove this command for no title

	% \tableofcontents % Output the table of contents (all sections on one slide)
	\tableofcontents[pausesections] % Output the table of contents (break sections up across separate slides)
\end{frame}

%----------------------------------------------------------------------------------------
%    PRESENTATION BODY SLIDES
%----------------------------------------------------------------------------------------

\section{Motivation}

%------------------------------------------------

\begin{frame}
	\frametitle{Motivation}
\end{frame}

%------------------------------------------------

\section{Modal logics}

%------------------------------------------------

\begin{frame}
	\frametitle{Normal modal logics}

		Syntax:
			\[ \varphi := \top
					\mid \bot
					\mid \varphi \lor \varphi
					\mid \varphi \land \varphi
					\mid \varphi \imp \varphi
					\mid \neg \varphi
					\mid \Dia \varphi
					\mid \Box \varphi \]

		\bigskip

		A \emph{normal modal logic} is a set of formulas of the above form containing K and Dual and closed under \emph{modus ponens}, \emph{uniform substitution}, and \emph{necessitation}.

		\bigskip

		(K)\hspace{1.08cm}$\Box (p \imp q) \imp (\Box p \imp \Box q)$\\
		(Dual)\hspace{.5cm}$\Box p \leftrightarrow \neg \Dia \neg p$

		\bigskip

		The smallest such logic is denoted by $\mathbf{K}$.

		\bigskip

		The logic \textbf{S4} is defined as $\mathbf{K} \union \set{\Box p \imp \Box \Box p, \Box p \imp p}$

\end{frame}


%------------------------------------------------

\section{Kripke models}

%------------------------------------------------

\begin{frame}
	\frametitle{Relational semantics for modal logic}

	A \emph{Kripke frame} is a tuple $(X, R)$ where $X$ is a set and $R \sub X \times X$.

	\bigskip

	A \emph{Kripke model} is a triple $(X, R, V)$ where $(X, R)$ is a Kripke frame and $V: \mathbf{Prop} \to \pset{X}$.

	\bigskip

	A \emph{pointed Kripke model} is a 4-tuple $(X, R, V, x)$ where $(X, R, V)$ is a Kripke model and $x \in X$.

	\bigskip

	Key semantic definition:
		\[ (X, R, V, x) \models \Box \varphi \miff (\forall x' \in X)(xRx' \mimp (X, R, V, x') \models \varphi) \]

\end{frame}

%------------------------------------------------

\begin{frame}
	\frametitle{The class \textbf{Pre}}

	A \emph{pre-order} is a tuple $(X, R)$ such that the following hold for all $x, y, z \in X$.
	\begin{itemize}
		\item $xRx$
		\item $xRy$ and $yRz$ implies $xRz$
	\end{itemize}

	\bigskip

	The class of pre-orders is denoted by $\mathbf{Pre}$.

	\bigskip

	Fact: The logic $\mathbf{S4}$ is sound and complete with respect to the class of frames $\mathbf{Pre}$
	\begin{align*}
		\varphi \in \mathbf{S4} &\miff \mathbf{Pre} \models \varphi \\
		                        &\miff (\forall (X, R) \in \mathbf{Pre})(\forall V \in \pset{X}^{\mathbf{Prop}})(\forall x \in X)\Bl((X, R, V, x) \models \varphi\Br)
	\end{align*}

\end{frame}

%------------------------------------------------

\section{Topology}

%------------------------------------------------

\begin{frame}
	\frametitle{Basic topology}

	A \emph{topological space} is a tuple $(X, \tau)$ where $X$ is a set and $\tau \sub \pset{X}$ where $\tau$ satisfies the following.
	\begin{itemize}
		\item $\empty, X \in \tau$
		\item $S \sub \tau$ and $|S| < \omega$ implies $\biginter S \in \tau$
		\item $S \sub \tau$ implies $\bigunion S \in \tau$
	\end{itemize}

	\bigskip

	A set $S$ is \emph{open} if $S \in \tau$, \emph{closed} if $X - S \in \tau$, and \emph{clopen} if it is both open and closed.

\end{frame}

%------------------------------------------------

\begin{frame}
	\frametitle{Basic topology cont.}

	Given a subset $S \sub X$, the \emph{interior of $S$}, denoted by $\text{int}(S)$, is the largest open subset of $S$, or, equivalently,
		\[ \bigunion \compin{U}{\tau}{U \sub S} \]
	The \emph{closure of $S$}, denoted by $\text{cl}(S)$, is the smallest closed superset of $S$, or, equivalently,
		\[ \biginter \compsub{C}{X}{ X-C \in \tau \text{ and } S \sub C} \]

\end{frame}

%------------------------------------------------

\begin{frame}
	\frametitle{The class \textbf{Alx}}

	Recall that a topospace $(X, \tau)$ satisfies the following.
	\begin{itemize}
		\item $\empty, X \in \tau$
		\item $S \sub \tau$ and $|S| < \omega$ implies $\biginter S \in \tau$
		\item $S \sub \tau$ implies $\bigunion S \in \tau$
	\end{itemize}

	\bigskip

	A topospace is called \emph{Alexandrov} if it also satisfies the following \emph{strengthening} of the second requirement above.
	\begin{itemize}
		\item $S \sub \tau$ \xcancel{and $|S| < \omega$} implies $\biginter S \in \tau$
	\end{itemize}

	\bigskip

	The class of Alexandrov topospaces is denoted by $\mathbf{Alx}$.

\end{frame}

%------------------------------------------------

\section{Topomodels}

%------------------------------------------------

\begin{frame}
	\frametitle{Topological semantics for modal logic}

	A \emph{topomodel} is a triple $(X, \tau, V)$ where $(X, \tau)$ is a topospace and $V: \mathbf{Prop} \to \pset{X}$.

	\bigskip

	A \emph{pointed topomodel} is a 4-tuple $(X, \tau, V, x)$ where $(X, \tau, V)$ is a topomodel and $x \in x$.

	\bigskip

	Key semantic definition:
		\[ (X, \tau, V, x) \models \Box \varphi \miff (\exists U \in \tau)\Bl(x \in U \text{ and } (\forall y \in U)\Bl((X, \tau, V, y)\models \varphi\Br)\Br) \]

	\bigskip

	This implies that
		\[ [\![\Box \varphi]\!] = \text{int}\Bl([\![\varphi]\!]\Br)\]
		\[ [\![\Dia \varphi]\!] = \text{cl}\Bl([\![\varphi]\!]\Br)\]

\end{frame}

%------------------------------------------------

\section{Model conversion}

%------------------------------------------------

\begin{frame}
	\frametitle{The upset topology}

	Given a pre-order $\mathbf{X} := (X, R)$, an \emph{upset} is a subset $S \sub X$ such that
		\[ (\forall x \in X)(x \in S \text{ and } xRy \text{ implies } y \in S) \]

	\bigskip

	We denote that set of all upsets on $\mathbf{X}$ by $\text{Up}(\mathbf{X})$.

	\bigskip

	Observe that $(X, \text{Up}(\mathbf{X}))$ is an Alexandrov topospace, so given an \textbf{S4} Kripke model, we can create a topomodel satisfying the same theory.

\end{frame}

%------------------------------------------------

\begin{frame}
	\frametitle{The specialisation order}

	Given an topospace $\mathbf{X} := (X, \tau)$, we can define the \emph{specialisation order on $\mathbf{X}$} as follows:
	\begin{align*}
		xR_{\mathbf{X}}y &:\Longleftrightarrow y \in \text{Cl}(\set{x}) \\
		                 &~\Longleftrightarrow y \in \biginter \compsub{C}{X}{ X-C \in \tau \text{ and } \set{x} \sub C} \\
		                 &~\Longleftrightarrow y \in \biginter \compsub{C}{X}{ X-C \in \tau \text{ and } x \in C} \\
		                 &~\Longleftrightarrow (\forall C \sub X)(X-C \in \tau \text{ and } x \in C \mimp y \in C)
	\end{align*}

	\bigskip

	Observe that $R_{\mathbf{X}}$ is a pre-order, so given an topomodel, we can create an \textbf{S4} Kripke model satisfying the same theory.

\end{frame}

%------------------------------------------------

\section{Future work}

%------------------------------------------------

\begin{frame}
	\frametitle{Future work}
\end{frame}

%----------------------------------------------------------------------------------------
%    CLOSING SLIDE
%----------------------------------------------------------------------------------------

\begin{frame}[plain] % The optional argument 'plain' hides the headline and footline
	\begin{center}
		{\Huge The End}

		\bigskip\bigskip % Vertical whitespace

		{\LARGE Questions? Comments?}
	\end{center}
\end{frame}

%----------------------------------------------------------------------------------------

\end{document}
