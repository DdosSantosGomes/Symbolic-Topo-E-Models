\documentclass[12pt,a4paper]{article}
\usepackage{etex,datetime,setspace,latexsym,amssymb,amsmath,amsthm}
\usepackage{fancybox,dialogue,float,wrapfig,enumerate,microtype}
\usepackage{verbatim,xcolor,multicol,titlesec,tabularx,mdframed}

\usepackage[utf8]{inputenc}
\usepackage[pdftex]{hyperref}
\usepackage[margin=2cm,bottom=3cm,footskip=15mm]{geometry}
\parindent0cm
\parskip1em

\usepackage{tikz}
\usetikzlibrary{arrows,trees,positioning,shapes,patterns}
\usetikzlibrary{intersections,calc,fpu,decorations.pathreplacing}

\usepackage[T1]{fontenc} % better fonts

% Haskell code listings in our own style
\usepackage{listings,color}
\definecolor{lightgrey}{gray}{0.35}
\definecolor{darkgrey}{gray}{0.20}
\definecolor{lightestyellow}{rgb}{1,1,0.92}
\definecolor{dkgreen}{rgb}{0,.2,0}
\definecolor{dkblue}{rgb}{0,0,.2}
\definecolor{dkyellow}{cmyk}{0,0,.7,.5}
\definecolor{lightgrey}{gray}{0.4}
\definecolor{gray}{gray}{0.50}
\lstset{
  language        = Haskell,
  basicstyle      = \scriptsize\ttfamily,
  keywordstyle    = \color{dkblue},     stringstyle     = \color{red},
  identifierstyle = \color{dkgreen},    commentstyle    = \color{gray},
  showspaces      = false,              showstringspaces= false,
  rulecolor       = \color{gray},       showtabs        = false,
  tabsize         = 8,                  breaklines      = true,
  xleftmargin     = 8pt,                xrightmargin    = 8pt,
  frame           = single,             stepnumber      = 1,
  aboveskip       = 2pt plus 1pt,
  belowskip       = 8pt plus 3pt
}
\lstnewenvironment{code}[0]{}{}
\lstnewenvironment{showCode}[0]{\lstset{numbers=none}}{} % only shown, not compiled

% will the real phi please stand up
\renewcommand{\phi}{\varphi}

% load hyperref as late as possible for compatibility
\usepackage[pdftex]{hyperref}
\hypersetup{
  pdfborder = {0 0 0},
  breaklinks = true,
  linktoc = all,
}
\pdfinfoomitdate=1
\pdftrailerid{}
\pdfsuppressptexinfo15

% Commands and aliases

\def\TODO{\textcolor{red}{\textbf{Finish me.}}}
\def\FIXME{\textcolor{red}{\textbf{Fix me.}}}
\def\reff{\textcolor{red}{\textbf{?}}}
% Fonts
\def\Bl{\Bigl}
\def\Br{\Bigr}
% Sets
\def\contains{\ni}
\def\notcontains{\notbin{\ni}}
\def\empty{\varnothing}
\def\sub{\subseteq}
\def\sup{\supseteq}
\def\notsub{\notbin{\sub}}
\def\notsup{\notbin{\sup}}
\def\union{\cup}
\def\inter{\cap}
\def\bigunion{\bigcup}
\def\biginter{\bigcap}
\def\dunion{\sqcup}
\newcommand{\pset}[1]{\wp(#1)}
\newcommand{\set}[1]{\{#1\}}
\newcommand{\indset}[2]{\set{#1}_{#2}}
\newcommand{\bindset}[3]{\set{#1}_{#2}^{#3}}
\newcommand{\comp}[2]{\{ #1 \mid #2 \}}
\newcommand{\compin}[3]{\{ #1 \in #2 \mid #3 \}}
\newcommand{\compsub}[3]{\{ #1 \sub #2 \mid #3 \}}
\newcommand{\quo}[2]{{#1}_{\nicefrac{}{#2}}}
% Functions
\def\too{\longrightarrow}
\def\mapstoo{\longmapsto}
\def\after{\circ}
\def\inject{\xhookrightarrow{}}
% General logic
\def\imp{\rightarrow}
\def\bigland{\bigwedge}
\def\biglor{\bigvee}
\def\refutes{\notbin{\models}}
\def\seqimp{\Rightarrow}
\def\iso{\cong}
\def\notiso{\notbin{\simeq}}
\def\substruct{\mathbin{\trianglelefteqslant}}
\def\notsubstruct{\notbin{\substruct}}
\def\proves{\vdash}
\newcommand{\provable}[2]{\proves^{#1}_{#2}}
% Modal logic
\def\Dia{\Diamond}
\def\HalfDiamondLeft{\ensuremath{\clipbox{0pt 0pt 4pt 0pt}{$\Diamond$}}}
\def\HalfDiamondRight{\ensuremath{\clipbox{4pt 0pt 0pt 0pt}{$\Diamond$}}}
\def\HalfBoxLeft{\ensuremath{\clipbox{0pt 0pt 4pt 0pt}{$\square$}}}
\def\HalfBoxRight{\ensuremath{\clipbox{4pt 0pt 0pt 0pt}{$\square$}}}
\def\Dox{\HalfDiamondLeft\raisebox{-1.4pt}{\scalebox{1.2}{\HalfBoxRight}}}
\def\Bia{\raisebox{-1.4pt}{\scalebox{1.2}{\HalfBoxLeft}}\HalfDiamondRight}
\def\mcbox{\mathrlap{\Box}{\hspace{2.3pt}\raisebox{1.8pt}{\scalebox{0.8}{$*$}}}\hspace{2.2pt}}
\def\mcdia{\mathrlap{\Diamond}{\hspace{1.6pt}\raisebox{1.2pt}{\scalebox{0.8}{$*$}}}\hspace{1.5pt}}
% Orders
\def\leq{\leqslant}
\def\geq{\geqslant}
\def\notleq{\notbin{\leq}}
\def\notgeq{\notbin{\geq}}
\def\notlt{\notbin{<}}
\def\notgt{\notbin{>}}
\def\join{\vee}
\def\meet{\wedge}
\def\bigjoin{\bigvee}
\def\bigmeet{\bigwedge}
\newcommand{\upset}[1]{{\uparrow}#1}
\newcommand{\downset}[1]{{\downarrow}#1}
\newcommand{\Upset}[1]{{\Uparrow}#1}
\newcommand{\Downset}[1]{{\Downarrow}#1}
% Numbers
\def\NN{\mathbb{N}}
\def\QQ{\mathbb{Q}}
\def\RR{\mathbb{R}}
% Topology
\def\interior{\text{int}}
\newcommand{\closure}[1]{\overline{#1}}
% Proof-writing
\def\subsubqed{\hfill $\checkmark$}
\def\subqed{\hfill $\square$}
\def\qed{\hfill $\blacksquare$}
\def\mimp{\Rightarrow}
\def\miff{\Leftrightarrow}
\def\defiff{\ :\Longleftrightarrow}
\def\awtbs{as was to be shown}
\def\wlogmimp{\xRightarrow{\text{WLOG}}}
\def\LHS{\text{LHS}}
\def\RHS{\text{RHS}}
\def\Nte{\emph{Note - }}
\newcommand{\Rmk}[1]{\emph{Remark #1 -}}
\newcommand{\Def}[2]{\textbf{Definition #1} (#2) -}
\newcommand{\Exm}[1]{\textbf{Example #1} -}
\newcommand{\Thm}[1]{\textbf{Theorem #1} -}
\newcommand{\Lem}[1]{\textbf{Lemma #1} -}
\newcommand{\Fct}[1]{\textbf{Fact #1} -}
\newcommand{\Obs}[1]{\textbf{Observation #1} -}
\def\Proof{\emph{Proof - }}
% Project-specific
\def\XX{\mathbf{X}}


\title{My Report}
\author{Me}
\date{\today}
\hypersetup{pdfauthor={Me}, pdftitle={My Report}}

\begin{document}

\maketitle

\begin{abstract}
We give a toy example of a report in \emph{literate programming} style.
The main advantage of this is that source code and documentation can
be written and presented next to each other.
We use the listings package to typeset Haskell source code nicely.
\end{abstract}

\vfill

\tableofcontents

\clearpage

% We include one file for each section. The ones containing code should
% be called something.lhs and also mentioned in the .cabal file.


\section{How to use this?}

To generate the PDF, open \texttt{report.tex} in your favorite \LaTeX editor and compile.
Alternatively, you can manually do
\texttt{pdflatex report; bibtex report; pdflatex report; pdflatex report} in a terminal.

You should have stack installed (see \url{https://haskellstack.org/}) and
open a terminal in the same folder.

\begin{itemize}
  \item To compile everything: \verb|stack build|.
  \item To open ghci and play with your code: \verb|stack ghci|
  \item To run the executable from Section \ref{sec:Main}: \verb|stack build && stack exec myprogram|
  \item To run the tests from Section \ref{sec:simpletests}: \verb|stack clean && stack test --coverage|
\end{itemize}


\input{../lib/Basics.lhs}

\input{../exec/Main.lhs}

\input{../test/simpletests.lhs}


\section{Optional: Profiling}\label{sec:Profiling}

The GHC compiler comes with a profiling system to keep track of which
functions are executed how often and how much time and memory they take.
To activate this RTS, we compile and execute our program as follows:

\begin{verbatim}
stack clean
stack build --profile
stack exec --profile myprogram -- +RTS -p
\end{verbatim}

Results are saved in the file \texttt{myprogram.prof} which looks as follows.
Note for example, that funnyfunction was called on 14 entries.

\begin{small}
\begin{verbatim}
                                                 individual      inherited
COST CENTRE       MODULE          no. entries  %time %alloc   %time %alloc

MAIN              MAIN            227       0    0.0    0.8     0.0  100.0
 main             Main            455       0    0.0   29.6     0.0   38.6
  funnyfunction   Basics          518      14    0.0    0.6     0.0    0.6
  randomnumbers   Basics          464       0    0.0    0.3     0.0    8.4
   randomRIO      System.Random   468       0    0.0    0.0     0.0    8.0
   ...
\end{verbatim}
\end{small}

For many more RTS options, see the GHC documentation online at
\url{https://downloads.haskell.org/~ghc/latest/docs/html/users_guide/profiling.html}.



\section{Conclusion}\label{sec:Conclusion}

\textcolor{red}{\textbf{Finish me.}}


\addcontentsline{toc}{section}{Bibliography}
\bibliographystyle{alpha}
\bibliography{references.bib}

\end{document}
